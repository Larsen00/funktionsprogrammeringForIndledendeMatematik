%--Template Made by Daniel Ian Alexander Ramfeldt--30-09-2019
%----------------basic pakages----------------
\usepackage[utf8]{inputenc}                     % for special chareacter like "æøå"
\usepackage[danish]{babel}
\usepackage{csquotes}
\MakeOuterQuote{"}                      % for proper namning of autogenerated stuff
\usepackage[left=3.5cm,right=3.5cm,top=4cm,bottom=4cm]{geometry}  % for page size and margin setting
\usepackage{float}
\usepackage{calc}
\usepackage{hyperref}                           % page numbers and '\ref's become clickable
\usepackage{authblk}                            % allow a footnote style of author/affiliation
\usepackage{subfiles} 
\usepackage{parskip}                            % should remove indent at new lines
% \usepackage{cite}
\usepackage{enumitem}                           % for control of 'enumerate'numbering
\usepackage{listings}                           % for control of 'itemize spacing
\usepackage{pdfpages}
\usepackage{xcolor}
% \usepackage[framed,numbered,autolinebreaks,useliterate]{Opsætning/mcode}
\usepackage{datetime}

%-----------images manipulation----------------
\usepackage{subcaption}                        % for more pictures in one figure
\usepackage{graphicx}    
\usepackage{wrapfig}                           % Allows figures or tables to have text wrapped around them
\usepackage{lscape}
\usepackage{svg}                               % for the automated integration of SVG graphics
                                               %  for for image manipulation
\graphicspath{{/graphics}}
    


%--------------math packages------------------
\usepackage{amssymb}
\usepackage{amsmath}
\usepackage{amsthm}
\usepackage{mathtools}

%----------------tables-------------------------
\usepackage{colortbl}                           % colour in tables
\usepackage{makecell}                           % for  common tabular layouts 
\usepackage{longtable}
\usepackage{array}
\usepackage{tabu}
\usepackage{tabularx}
\usepackage{dcolumn}
\usepackage{multirow}
\usepackage{booktabs}                            % Enhances the quality of tables

%----------------header-&-footer----------------

\usepackage{csquotes}


\usepackage{fancyhdr}
\usepackage{lastpage}
\pagestyle{fancy}
\fancyhf{}
\lhead{}
\rhead{Funktionel Modellering af Matematiske Systemer i F\#}
\rfoot{Side \thepage{} af \pageref{LastPage}}     % page
\rfoot{Side \thepage{}} 
%\lfoot{\today}                                  % gives date 

%---------------------------------------------


\usepackage{tikz}
%\scalebox{<h-scale>}[<v-scale>]{<content>}

\usepackage{listings}
\usepackage{xcolor}
\usepackage{centernot}

% Definerer et nyt theorem-miljø
% Definerer en ny theorem-stil
\newtheoremstyle{break}  % navn
  {}                      % Plads over
  {}                      % Plads under
  {\itshape}              % Skrifttype i body
  {}                      % Indrykning
  {\bfseries}             % Skrifttype i theorem-hoved
  {.}                     % Punktuering efter theorem-hoved
  {\newline}              % Plads efter theorem-hoved
  {}                      % 'Theorem'-hoved specifikation (kan være efterladt tom, dvs. 'normal')

% Anvender theorem-stilen
\theoremstyle{break}
\newtheorem{theorem}{Sætning}
\newtheorem{egenskab}{Egenskab}
\newtheorem{definition}{Defination} % Definerer 'definition' miljøet

% -------------------------------------------
% så man kan lave sub sections 
\usepackage{titlesec}
\usepackage{hyperref}

\titleclass{\subsubsubsection}{straight}[\subsection]

\newcounter{subsubsubsection}[subsubsection]
\renewcommand\thesubsubsubsection{\thesubsubsection.\arabic{subsubsubsection}}
\renewcommand\theparagraph{\thesubsubsubsection.\arabic{paragraph}} % optional; useful if paragraphs are to be numbered

\titleformat{\subsubsubsection}
  {\normalfont\normalsize\bfseries}{\thesubsubsubsection}{1em}{}
\titlespacing*{\subsubsubsection}
{0pt}{3.25ex plus 1ex minus .2ex}{1.5ex plus .2ex}

\makeatletter
\renewcommand\paragraph{\@startsection{paragraph}{5}{\z@}%
  {3.25ex \@plus1ex \@minus.2ex}%
  {-1em}%
  {\normalfont\normalsize\bfseries}}
\renewcommand\subparagraph{\@startsection{subparagraph}{6}{\parindent}%
  {3.25ex \@plus1ex \@minus .2ex}%
  {-1em}%
  {\normalfont\normalsize\bfseries}}
\def\toclevel@subsubsubsection{4}
\def\toclevel@paragraph{5}
\def\toclevel@paragraph{6}
\def\l@subsubsubsection{\@dottedtocline{4}{7em}{4em}}
\def\l@paragraph{\@dottedtocline{5}{10em}{5em}}
\def\l@subparagraph{\@dottedtocline{6}{14em}{6em}}
\makeatother


% --------------------------------

\definecolor{bluekeywords}{rgb}{0.13, 0.13, 1}
\definecolor{greencomments}{rgb}{0, 0.5, 0}
\definecolor{red}{rgb}{0.9, 0, 0.6}
\definecolor{liteblue}{rgb}{0.28,0.46,1.0}

\definecolor{codegreen}{rgb}{0,0.6,0}
\definecolor{codegray}{rgb}{0.5,0.5,0.5}
\definecolor{codepurple}{rgb}{0.58,0,0.82}
\definecolor{backcolour}{rgb}{0.95,0.95,0.92}


\definecolor{bluekeywords}{rgb}{0.13, 0.13, 1}
\definecolor{purplekeywords}{rgb}{0.58, 0, 0.82}

\lstdefinelanguage{FSharp}%
{
  morekeywords={when, or, elif, new, match, with, rec, open, module, namespace, % 
    and, for, while, true, false, in, do, begin, end, fun, function, return, yield, try, %
    mutable, if, then, else, cloud, async, use, abstract, interface, inherit, finally,
    let!, return!, do!, yield!, use!, from, select, where, by, val },
  keywordstyle=\color{codepurple},
  sensitive=true,
  basicstyle=\ttfamily\footnotesize,
  breaklines=true,
  xleftmargin=\parindent,
  aboveskip=\bigskipamount,
  tabsize=1,
  morecomment=[l][\color{greencomments}]{///},
  morecomment=[l][\color{greencomments}]{//},
  morecomment=[l][\color{greencomments}]{\#},
  morestring=[b]",
  showstringspaces=false,
  literate={`}{\`}1,
  stringstyle=\color{greencomments},
  emph={[2]let, rec, def, raise, member, static, type, of, }, % Let og var er blå
  emphstyle={[2]\color{bluekeywords}},
  %emph={[3]Rational, rational, Number, int, Int}, % Typer
  emphstyle={[3]\color{liteblue}},
  emph={[3]Mul, Add, N, Sub, X, Div, Neg, Int, Rational, Complex, C, R},
  emph={
    zero, one, two, isZero, isOne, isNegative, abs, greaterThan, tryMakeInt, toString, sumList, factorial, failwith, ValueError, make, equal, posetive, isInt, makeRatInt, absRational, propositional_formula, neg, operation, makeRational, eval, add, mul, sub, neg, dimMatrix, matrixValidMajor, matrixVectorLength, scalarVector, scalarMatrix, map, orthogonalBacis, Gram_Schmidt, sumProj, not, vectorOf, proj, extendMatrix, correctOrder, corectOrderCheck, map2,vectorMulElementWise, conjugateVector, innerProduct, conjugate, fold, length, performRowOperationGen, multipleRowOperationsGen, getIndependetBacisGen, empty, getBacismatrixGen, choose, standardBacis, oneof, rowOperation, isOrthogonalBacis, gramSchmidtIsOrthogonal, classify, addVector, subVector, addMatrix, giveMatrixHaveSameOrder, sumRows, matrixToVector, matrix, matrixMulVector, matrixProduct, numberGen, vectorGen, matrixGen, parenthesis, ExpressionToInfix,infixExpression, string, etf, simplifyExpr, absNumber, div, applyCommutative, isAdd, isMul, isDiv, isSub, isNeg, simplifyOperation, diff, tryReduce, inv, mulComplex, complexDivRational, mulConjugate, divComplex, expressionOnX, isolateX, containsX, getNumber, getVariable
    }, % Dit funktionsnavn her (purple)
  emphstyle=\color{red},
  xleftmargin=2em,
}




\lstdefinestyle{fsharpstyle}{
   backgroundcolor=\color{gray!10},
   commentstyle=\color{codegreen},
   keywordstyle=\color{codepurple},
   numberstyle=\tiny\color{codegray},
   stringstyle=\color{red},
   basicstyle=\ttfamily,
   breakatwhitespace=false,
   breaklines=true,
   captionpos=b,
   keepspaces=true,
   numbers=left,
   numbersep=5pt,
   showspaces=false,
   showstringspaces=false,
   showtabs=false,
   tabsize=2,
   frame=none
}

\lstset{style=fsharpstyle}

\lstdefinestyle{output}{
    backgroundcolor=\color{gray!10},  % Set the background color
    literate={\%}{{\%}}1,              % Properly display percent symbols
    basicstyle=\ttfamily\small,        % Use a monospaced font
    numbers=none,                      % No line numbers
    frame=none,                        % No frame around the code
}
