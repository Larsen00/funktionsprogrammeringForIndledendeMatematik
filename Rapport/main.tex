\documentclass{article}
\usepackage[utf8]{inputenc}
\usepackage{pdfpages}
\input{Opsætning/Pakker}


\begin{document}

\begin{titlepage} % Suppresses displaying the page number on the title page and the subsequent page counts as page 1
	\newcommand{\HRule}{\rule{\linewidth}{0.5mm}} % Defines a new command for horizontal lines, change thickness here
	
	\center % Centre everything on the page
	%------------------------------------------------
	%	READ ME
	%------------------------------------------------
	   
	    %This version of the title page contains both signatures and pictures of participants, this version is best suited for larger reports and or projects. 
	
	%------------------------------------------------
	%	Headings
	%------------------------------------------------
	
	\textsc{\LARGE Danmarks Tekniske Universitet}\\[1.5cm] % Main heading such as the name of your university/college
	
    \includegraphics[scale=0.15]{Opsætning/DTULogo.png}\\
	%\textsc{\Large Major Heading}\\[0.5cm] % Major heading such as course name
	
	%\textsc{\large Minor Heading}\\[0.5cm] % Minor heading such as course title
	
	%------------------------------------------------
	%	Title
	%------------------------------------------------
	
	\HRule\\[0.5cm]
	
	{\huge\bfseries Bachelorprojekt}\\[0.4cm] % Title of your document

	\HRule\\[0.5cm]
	%\textsc{\Large Danmarks Tekniske Universitet}\\[0.5cm] % Major heading such as course name
	
	\textsc{\Large Funktionel Modellering af Matematiske Systemer i F\#}\\[1cm] % Minor heading such as course title
	
	%------------------------------------------------
	%	Author(s)
	%------------------------------------------------
    \vfill\vfill\vfill
    \begin{minipage}{\textwidth}
		\begin{flushleft}
            \centering

            Jonas Dahl Larsen (s205829)
            

		\end{flushleft}
	 \end{minipage} 
    % \\[1cm]
    % \vfill \vfill
    \vspace*{1\baselineskip}


    


	%------------------------------------------------
	%	Date
	%------------------------------------------------
	
	%\vfill\vfill\vfill % Position the date 3/4 down the remaining page
	
	{\large \today} % Date, change the \today to a set date if you want to be precise
	
	\vfill % Push the date up 1/4 of the remaining page
	
\end{titlepage}
\tableofcontents
\newpage
\section{Introduktion}


I 2023 valgte Danmarks Tekniske Universitet at anvende Python som et hjælpeværktøj i deres grundlæggende matematikkursus "01001 Matematik 1a (Polyteknisk grundlag)". Python er et af de mest anvendte programmeringssprog \footnote{https://www.statista.com/statistics/793628/worldwide-developer-survey-most-used-languages/}, kun overgået af to sprog, der primært bruges sammen til at udvikle hjemmesider. Derfor har Python, med sit dynamiske skrevne sprog og en række matematiske programudvidelser som SymPy \footnote{https://www.sympy.org/en/index.html}, været et oplagt valg som programmeringssprog til det grundlæggende matematikkursus tilbudt af DTU.

Projektet vil undersøge, hvordan et funktionsprogrammeringssprog, kan gavne de studerendes forståelse af de grundlæggende matematiske koncepter. Formålet er at guide læseren gennem opbygningen af en række funktionsprogrammer baseret på grundlæggende universitetsmatematik og dermed illustrere anvendelser. Projektet beskriver en generel struktur til opbygning og anvendelse af et funktions programmeringsprogram. Der tages udgangspunkt i F\# \footnote{https://en.wikipedia.org/wiki/F\_Sharp\_(programming\_language)}, men beskrivelserne af programmerne vil også kunne anvendes i lignende funktionsprogrammeringssprog.

Rapporten begynder med at forklare nogle Fundamentale koncepter inden for funktionsprogrammering samt metoder til validering af programmerne. 


\section{Fundamentale koncepter}
\subsection{Introduktion til Funktions Programmering}
Det forventes, at læseren har kendskab til programmering. Der gives derfor kun en kort beskrivelse af syntaks og notation, så læsere, der ikke er bekendt med F\#, kan forstå de eksempler, der løbende vil forekomme i rapporten. Vi begynder derfor med at betragte funktionen for fakultet Ligning \eqref{Fakultet}.

\begin{equation}
    \label{Fakultet}
    f(n) = \begin{cases} 
            1 &  n = 0  \\
            n \cdot f(n-1) & n > 0 \\
            \text{undefined} & n < 0 
           \end{cases}
\end{equation}

Et eksempel på en implementering af \eqref{Fakultet} i F\# er givet i Listing \ref{lst:fsharp_factorial}.
23

\begin{lstlisting}[language={FSharp}, label={lst:fsharp_factorial}, caption={Eksempel på Fakultet i F\#}]
// Fakultet i F#
let rec factorial n =
    match n with
    | 0             -> 1 
    | x when x > 0  -> x * factorial (x - 1)
    | _             -> failwith "Negative argument"
\end{lstlisting}

I F\# anvendes \textcolor{blue}{let} til at definere en ny variabel eller, i dette tilfælde, en funktion kaldet \textcolor{red}{factorial}. Næste nøgleord er \textcolor{blue}{rec}, hvilket indikerer, at funktionen er rekursiv. Funktionen tager et argument \(n\), og i linje 3 starter et match-udtryk. Her er \(n\) vores udtryk, og efter \textcolor{codepurple}{with} begynder en række mønstre. Med tilhørende udtryk, der svare til de tre tilfælde i \eqref{Fakultet}. 
    
I F\#, er det som udgangspunkt ikke nødvendigt at anvende parenteser som i andre programmeringssprog. Derfor vil de kun blive anvendt, hvor det er nødvendigt gennem rapporten, typisk i sammenhænge med kædning af funktioner. For at undgå brugen af parenteser kan man i F\# benytte pipe-operatorerne, $|>$ og $<|$, som fører resultatet fra en udledning direkte ind i den næste funktion. Nedenstående eksempel viser tre ækvivalente udtryk, der demonstrerer anvendelsen af disse operatorer.

\begin{lstlisting}[style=output, label={lst:pipe_operator}, caption={Eksempel på anvendelse af pipe-operatorer i F\# ved udregning af $(3!)! = 6! = 720 $.}]
> factorial (factorial 3);;
val it: int = 720

> factorial <| factorial 3;;
val it: int = 720

> factorial 3 |> factorial;;
val it: int = 720
\end{lstlisting}

Vi kan beskrive evalueringen af udtrykket i Listing \ref{lst:pipe_operator} som følgende, hvor $e_1 \leadsto e_2$ betyder, at $e_1$ evalueres til $e_2$:
\[
\begin{aligned}
&\text{factorial}(\text{factorial}(3)) \\
&\;\;\leadsto \text{factorial}(3 \times \text{factorial}(3 - 1)) \\
&\;\;\leadsto \text{factorial}(3 \times 2 \times \text{factorial}(2 - 1)) \\
&\;\;\leadsto \text{factorial}(3 \times 2 \times 1 \times \text{factorial}(1 - 1)) \\
&\;\;\leadsto \text{factorial}(3 \times 2 \times 1 \times 1) \\
&\;\;\leadsto \text{factorial}(6) \\
&\;\;\leadsto 6 \times \text{factorial}(6 - 1) \\
&\;\;\leadsto 6 \times 5 \times \text{factorial}(5 - 1) \\
&\;\;\leadsto 6 \times 5 \times 4 \times \text{factorial}(4 - 1) \\
&\;\;\leadsto 6 \times 5 \times 4 \times 3 \times \text{factorial}(3 - 1) \\
&\;\;\leadsto 6 \times 5 \times 4 \times 3 \times 2 \times \text{factorial}(2 - 1) \\
&\;\;\leadsto 6 \times 5 \times 4 \times 3 \times 2 \times 1 \times \text{factorial}(1 - 1) \\
&\;\;\leadsto 6 \times 5 \times 4 \times 3 \times 2 \times 1 \times 1 \\
&\;\;\leadsto 720
\end{aligned}
\]


\subsection{Typer}
% I F\#, i modsætning til Python, er typer tildelt ved kompileringstidspunktet, ikke under kørsel. Alle udtryk, inklusiv funktioner, har en defineret type. Typen for funktionen i Listing \textcolor{red}{\ref{lst:fsharp_factorial}} er $int \rightarrow int$. Det betyder, at det ikke er muligt at kalde funktionen med et argument, der ikke er af typen $int$. Typen for funktionen beskrives som $Factorial: int \rightarrow int$. Det ses dermed tydeligt hvordan f\# benytter notationen for afbildninger i mattematik, da den matematiske funktion for fakultet er en afbildning $f: \mathbb{Z} \to \mathbb{Z}$. Vi kan derfor formulere følgende omkring typer\footcitetitle[14]{HansenRischelFSharp}:
% \begin{gather*}
%     f: T_1 \rightarrow T_2 \\
%     f(e) : T_2 \iff e : T_1
% \end{gather*}
% % => f(e) ville give en fejl, hvis ikke e er af typen T1
% % <= per definition, så er f(e) : T2, hvis e : T1

% Hvis en funktion kaldes med et argument, der ikke matcher funktionens type, genereres en fejlmeddelelse. Derudover kan en type også bestå af en tuple af typer:
% \begin{gather*}
%     f: T_1 * T_2 * .. * T_n \rightarrow T_{n+1}\\
%     f (e_1, e_2, .., e_n) :T_{n+1} \iff e_1 : T_1 \land e_2 : T_2 \land .. \land e_n : T_n
% \end{gather*}
% En tuple, der kun består af to typer, kaldes et par. Givet en funktion $g: T_1 \rightarrow T_2 \rightarrow T_3$, betyder dette, at den tager et udtryk af typen $T_1$, som giver en funktion af typen $T_2 \rightarrow T_3$, hvor evalueringen af funktionen resulterer i $T_3$. Som eksempel på dette kan vi definere en multivariable funktion $f(x, y) = \sqrt{x! + y!}$ som er en afbildning $f: \mathbb{Z}^2 \to \mathbb{R}$ samt $g(y) = f(3, y) = \sqrt{3! + y!}$ som afbilder $g: \mathbb{Z} \to \mathbb{R}$, de tilsvarende F\# funktion kan defineres på følgende måder:
I F\# har alle udtryk, inklusiv funktioner, en defineret type. Typen for funktionen i Listing \ref{lst:fsharp_factorial} er $int \rightarrow int$. Det betyder, at det ikke er muligt at kalde funktionen med et argument, der ikke er af typen $int$.

I matematik kan vi beskrive funktioner som afbildninger ved brug af domæner og co-domæner på følgende måde:
\begin{gather*}
    f: \mathbb{F} \to \mathbb{F} \\
    g: \mathbb{F}^n \to \mathbb{F}
\end{gather*}
De tilsvarende funktioner i F\# vil have typerne:
\begin{gather*}
    f: float \to float \\
    g: float * float * \ldots * float \to float
\end{gather*}
Hvor typekonstruktøren $*$ svarer til det kartesiske produkt, men kalder det for en tuple. I begge tilfælde skal senere definitioner og funktionsanvendelser overholde disse beskrivelser. 

% Derudover kan en type også bestå af en tuple af typer:
% \begin{gather*}
%     f: T_1 * T_2 * \ldots * T_n \rightarrow T_{n+1} \\
%     f(e_1, e_2, \ldots, e_n) : T_{n+1} \iff e_1 : T_1 \land e_2 : T_2 \land \ldots \land e_n : T_n
% \end{gather*}
% En tuple, der kun består af to typer, kaldes et par. Givet en funktion $g: T_1 \rightarrow T_2 \rightarrow T_3$, betyder dette, at den tager et udtryk af typen $T_1$, som giver en funktion af typen $T_2 \rightarrow T_3$, hvor evalueringen af funktionen resulterer i $T_3$. Som eksempel på dette kan vi definere en multivariable funktion $f(x, y) = \sqrt{x! + y!}$ som er en afbildning $f: \mathbb{Z}^2 \to \mathbb{R}$ samt $g(y) = f(3, y) = \sqrt{3! + y!}$, som afbilder $g: \mathbb{Z} \to \mathbb{R}$. De tilsvarende F\# funktioner kan defineres på følgende måder:

% \begin{lstlisting}[language={FSharp}, label={lst:multivariable_function}, caption={Eksempel på typerne for en multivariable funktion i F\#}]
% // f: int -> int -> float
% let f x y = sqrt <| float(factorial x + factorial y)

% // g: int -> float
% let g = f 3
% \end{lstlisting}

Hvis en funktion kaldes med et argument, der ikke matcher funktionens type, genereres en fejlmeddelelse. For F\#'s vedkommende kontrolleres dette på oversættelsestidspunktet, hvilket er en af de store forskelle mellem F\# og Python, hvor typerne først kontrolleres under kørslen.





\subsection{Signatur filer og implementerings filer}
En standard F\# fil er lavet med .fs extension, denne fil indeholder alt den kode som er nødigt for at kunne køre programmet. En implementeringsfil kan have en signaturfil med .fsi extension, denne fil indeholder en beskrivelse af de typer og funktioner i implementeringsfilen som er tilgængelige for andre filer. En signaturfil kan derfor bruges som et blueprint for andre der ønsker at anvende eller replicere implementeringsfilen. I andre programmerings sprog vil man anse funktionerne i signaturfilen, som værende de funktioner der er tilgængelige ved åbning af modulet.

\subsection{Overskrivning af operatorer}
I F\# er det muligt at overskrive standardoperatorer, så de kan anvendes på egne typer. Denne teknik vil blive benyttet igennem rapporten til at definere matematiske operationer for de typer, vi udvikler.

\subsection{Property Based Testing}
Property Based Test (PBT) er en teknik til at teste korrekthed af egenskaber som man ved altid skal være opfyldt. Ved PBT genereres en række tilfældige input til en funktion, hvorefter det kontrolleres, om en given egenskab holder. Fokusset ved PBT er de fundamentale egenskaber, som en funktion skal overholde, og ikke de specifikke tilfælde som ved eksempelvis unit tests\footcitetitle{unit_test}.

På DTU lærer de studerende først om logik, hvor det introduceres, at en udsagnslogisk formel er gyldig (en tautologi), hvis den altid er sand. Der findes mange teknikker til at påvise gyldigheden af en udsagnslogisk formel. I "01001 Matematik 1a" på DTU lærer man at anvende sandhedstabeller, som demonstrerer gyldigheden af en formel. Eksempelvis vises hvordan \eqref{udsagnslogik} er gyldig.
\begin{gather}
    P \land (Q \land R) \iff (P \land Q) \land R
    \label{udsagnslogik}
\end{gather}

Vi kan også bruge PBT til at undersøge, om \eqref{udsagnslogik} holder, ved at definere egenskaben som en funktion af $P, Q$ og $R$, som vist i Listing \ref{valid1}.

\lstinputlisting[
    language=FSharp,
    label={valid1},
    caption={PBT af ligning \eqref{udsagnslogik}. Begge sider er omgivet af parenteser da $=$ har en højere præcedens end $\&\&$}
]{exampleCodes/valid1.fsx}

\begin{lstlisting}[style=output, label={lst:output_example}, caption={Output ved PBT af (\ref{udsagnslogik})}]
> let _ = Check.Quick propositional_formula;;
Ok, passed 100 tests.
\end{lstlisting}

Check.Quick er en del af "FsCheck"-biblioteket. Den tager en funktion som argument og genererer en række tilfældige input til funktionen på baggrund af funktionens type. Hvis funktionen returnerer "true" for alle input, vil testen lykkes. Hvis funktionen returnerer "false" for et input, vil testen fejle og give et eksempel på et input, der fejlede. I Listing \ref{valid1} er der anvendt "Check.Quick" til at teste, om \eqref{udsagnslogik} er gyldig. Funktionen "Check.Quick" returnerer "Ok, passed 100 tests," hvilket indikerer, at (\ref{udsagnslogik}) er gyldig. Det er vigtigt her at forstå, at dette ikke er det samme som at bevise, at \eqref{udsagnslogik} er gyldig, da ikke alle muligheder er blevet testet.

I nogle tilfælde vil det være en fordel at opskrive en PBT før implementeringen af en funktion, som man ved skal overholde en egenskab. På den måde anvendes Test Driven Development (TDD)\footcitetitle{TDD} til at teste, om ens egenskab forbliver overholdt under implementeringen. I dette projekt vil vi anvende PBT til at validere, at de matematiske egenskaber bliver overholdt af programmet.


\section{Symbolske lignings udtryk}
Det ønskes at kunne repræsentere simple ligninger som en type i F\#. Vil derfor gennemgå en del teori og funktion som er nødvendige for at kunne dette. Det vil give os et grundlæggende fundament for at kunne udøføre matematiske evalueringer som differentiering i F\#. Som de fleste andre programmer har F\# kun float og int som kan repræsentere tal. Derfor vil vi begynde med at definere et mondul som indeholder en type for tal. Tanke gangen her at gennemgå en opbygning af en måde at kunne repræsentere ligninger samt simplificere dem. Vi begrænset os selv til at kun have matematiske operationer som addition, subtraktion, negation, multiplikation og division.

\subsection{Tal Mondulet}
Vi begynder med opbygningen af et mondul som kan repræsentere tal. Typen for tal, består af tre konstruktører, for henholdvis heltal, rationale tal og komplekse tal. Dog er mondulet lavet med henblik på at kunne udvides med flere typer af tal. Måden resten af programmet på gør de eneste grav til tal er at der er definerede matematiske operationer i form af addition, subtraktion, negation, multiplikation og division. Samt at tallet inden for addition og multiplikation er associative. Dette gælder blandt andet ikke for en vektor, derfor vil vi senere betragte at udvide programmet med en type for vektorer. En udvidelse kunne være for reele tal, som kan håndtere "floating point errors"\footnote{\url{https://en.wikipedia.org/wiki/Floating-point_error_mitigation}}, men for ikke at komplikere programmet vil vi i denne opgave ikke betragte floats.  

\subsubsection{Rationelle tal Mondul}
Repræsentationen af rationale tal kan laves ved hjælp af danne et par af integers, hvor den ene integer er tælleren og den anden er nævneren. 

\begin{lstlisting}[language={FSharp}, label={lst:fsharp_factorial}, caption={Typen for rationelle tal}]
type rational = R of int * int
\end{lstlisting}

Nedestående er der givet en signatur fil for rational mondulet \ref{rational_fsi}. i Implementerings filen overloades de matematiske operatorer, ved hjælp af de klassiske regneregler for brøker\footnote{\url{https://en.wikipedia.org/wiki/Rational_number}}. 
\lstinputlisting[
    language=FSharp,
    label={rational_fsi},
    caption={Signatur filen for rational mondulet}
]{../modules/rational.fsi}

For forholdvis nemt at kunne sammenligne, men også for nemmere at undgå for store brøker, vil alle rationelle tal blive reduceret til deres simpleste from. Dette kan gøres ved at finde den største fælles divisor (GCD) \footnote{\url{https://en.wikipedia.org/wiki/Greatest_common_divisor}}. Der udover er det vigtigt at være opmærksom på man ikke foretager nul division. Derfor vil implementerings filen kaste en "System.DivideByZeroException" hvis nævneren er eller bliver nul. Signatur filen indeholder en række funktioner som bliver anvendt af andre filer.
 
\subsubsection{Komplekse tal Mondul}

\subsection{Ligningsudtryk Mondulet}
\subsubsection{Ligninger som træer}
\subsubsection{Infix og Prefix notation}
\subsubsection{Generering af Ligningsudtryk}

\subsection{Simplifikation af Ligningsudtryk}
\subsubsection{PBT af simplifikationen} % Beskriv hvordan man laver property based testen før simplifikationen

\subsection{differentiering af Ligningsudtryk}

\end{document}

