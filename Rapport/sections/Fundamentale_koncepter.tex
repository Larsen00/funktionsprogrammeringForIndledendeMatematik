
\section{Fundamentale koncepter}
Det forventes, at læseren har kendskab til programmering. Der gives derfor kun en kort beskrivelse af syntaks og notation, så læsere, der ikke er bekendt med F\#, kan forstå de eksempler, der løbende vil forekomme i rapporten. 

\subsection{Introduktion til Funktions Programmering}
F\# er en del af ML (Meta Language) familien af funktionsprogrammeringssprog, som bygger på "typed lambda calculus"\footcitetitle{lambda_typed}. "Typed lambda calculus" er en udvidelse af "lambda calculus"\footcitetitle{lambda}, som er et matematisk system. Det giver et formelt fundament for at definere og manipulere funktioner.

Vi begynder med at betragte funktionen for fakultet Ligning \eqref{Fakultet}.

\begin{equation}
    \label{Fakultet}
    f(n) = \begin{cases} 
            1 &  n = 0  \\
            n \cdot f(n-1) & n > 0 \\
            \text{undefined} & n < 0 
           \end{cases}
\end{equation}

Et eksempel på en implementering af Ligning \eqref{Fakultet} i F\# er givet i Listing \ref{lst:fsharp_factorial}.

\begin{lstlisting}[language={FSharp}, label={lst:fsharp_factorial}, caption={Eksempel på Fakultet i F\#}]
// Fakultet i F#
let rec factorial n =
    match n with
    | 0             -> 1 
    | x when x > 0  -> x * factorial (x - 1)
    | _             -> failwith "Negative argument"
\end{lstlisting}

I F\# anvendes \textcolor{blue}{let} til at definere en ny variabel eller, i dette tilfælde, en funktion kaldet \textcolor{red}{factorial}. Næste nøgleord er \textcolor{blue}{rec}, hvilket indikerer, at funktionen er rekursiv. Funktionen tager et argument \(n\), og i linje 3 starter et match-udtryk. Her er \(n\) vores udtryk, og efter \textcolor{codepurple}{with} begynder en række mønstre. Med tilhørende udtryk, der svare til de tre tilfælde i Ligning \eqref{Fakultet}. 
    
I F\#, er det som udgangspunkt ikke nødvendigt at anvende parenteser som i andre programmeringssprog. Derfor vil de kun blive anvendt, hvor det er passende gennem rapporten, typisk i sammenhænge med kædning af funktioner. For at undgå brugen af parenteser kan man i F\# benytte pipe-operatorerne, $|>$ og $<|$, som fører resultatet fra en udledning direkte ind i den næste funktion. Nedenstående eksempel viser tre ækvivalente udtryk, der demonstrerer anvendelsen af disse operatorer.

\begin{lstlisting}[style=output, label={lst:pipe_operator}, caption={Eksempel på anvendelse af pipe-operatorer i F\# ved udregning af $(3!)! = 6! = 720 $.}]
> factorial (factorial 3);;
val it: int = 720

> factorial <| factorial 3;;
val it: int = 720

> factorial 3 |> factorial;;
val it: int = 720
\end{lstlisting}

Vi kan beskrive evalueringen af udtrykket i Listing \ref{lst:pipe_operator} som følgende, hvor $e_1 \leadsto e_2$ betyder, at $e_1$ evalueres til $e_2$:
\[
\begin{aligned}
&\text{factorial}(\text{factorial}(3)) \\
&\;\;\leadsto \text{factorial}(3 \times \text{factorial}(3 - 1)) \\
&\;\;\leadsto \text{factorial}(3 \times 2 \times \text{factorial}(2 - 1)) \\
&\;\;\leadsto \text{factorial}(3 \times 2 \times 1 \times \text{factorial}(1 - 1)) \\
&\;\;\leadsto \text{factorial}(3 \times 2 \times 1 \times 1) \\
&\;\;\leadsto \text{factorial}(6) \\
&\;\;\leadsto 6 \times \text{factorial}(6 - 1) \\
&\;\;\leadsto 6 \times 5 \times \text{factorial}(5 - 1) \\
&\;\;\leadsto 6 \times 5 \times 4 \times \text{factorial}(4 - 1) \\
&\;\;\leadsto 6 \times 5 \times 4 \times 3 \times \text{factorial}(3 - 1) \\
&\;\;\leadsto 6 \times 5 \times 4 \times 3 \times 2 \times \text{factorial}(2 - 1) \\
&\;\;\leadsto 6 \times 5 \times 4 \times 3 \times 2 \times 1 \times \text{factorial}(1 - 1) \\
&\;\;\leadsto 6 \times 5 \times 4 \times 3 \times 2 \times 1 \times 1 \\
&\;\;\leadsto 720
\end{aligned}
\]


\subsection{Typer}
% I F\#, i modsætning til Python, er typer tildelt ved kompileringstidspunktet, ikke under kørsel. Alle udtryk, inklusiv funktioner, har en defineret type. Typen for funktionen i Listing \textcolor{red}{\ref{lst:fsharp_factorial}} er $int \rightarrow int$. Det betyder, at det ikke er muligt at kalde funktionen med et argument, der ikke er af typen $int$. Typen for funktionen beskrives som $Factorial: int \rightarrow int$. Det ses dermed tydeligt hvordan f\# benytter notationen for afbildninger i mattematik, da den matematiske funktion for fakultet er en afbildning $f: \mathbb{Z} \to \mathbb{Z}$. Vi kan derfor formulere følgende omkring typer\footcitetitle[14]{HansenRischelFSharp}:
% \begin{gather*}
%     f: T_1 \rightarrow T_2 \\
%     f(e) : T_2 \iff e : T_1
% \end{gather*}
% % => f(e) ville give en fejl, hvis ikke e er af typen T1
% % <= per definition, så er f(e) : T2, hvis e : T1

% Hvis en funktion kaldes med et argument, der ikke matcher funktionens type, genereres en fejlmeddelelse. Derudover kan en type også bestå af en tuple af typer:
% \begin{gather*}
%     f: T_1 * T_2 * .. * T_n \rightarrow T_{n+1}\\
%     f (e_1, e_2, .., e_n) :T_{n+1} \iff e_1 : T_1 \land e_2 : T_2 \land .. \land e_n : T_n
% \end{gather*}
% En tuple, der kun består af to typer, kaldes et par. Givet en funktion $g: T_1 \rightarrow T_2 \rightarrow T_3$, betyder dette, at den tager et udtryk af typen $T_1$, som giver en funktion af typen $T_2 \rightarrow T_3$, hvor evalueringen af funktionen resulterer i $T_3$. Som eksempel på dette kan vi definere en multivariable funktion $f(x, y) = \sqrt{x! + y!}$ som er en afbildning $f: \mathbb{Z}^2 \to \mathbb{R}$ samt $g(y) = f(3, y) = \sqrt{3! + y!}$ som afbilder $g: \mathbb{Z} \to \mathbb{R}$, de tilsvarende F\# funktion kan defineres på følgende måder:
I F\# har alle udtryk, inklusiv funktioner, en defineret type. Typen for funktionen i Listing \ref{lst:fsharp_factorial} er $int \rightarrow int$. Det betyder, at det ikke er muligt at kalde funktionen med et argument, der ikke er af typen $int$.

I matematik kan vi beskrive funktioner som afbildninger ved brug af domæner og co-domæner på følgende måde:
\begin{gather*}
    f: \mathbb{R} \to \mathbb{R} \\
    g: \mathbb{R}^n \to \mathbb{R}
\end{gather*}
De tilsvarende funktioner i F\# vil have typerne:
\begin{gather*}
    f: float \to float \\
    g: float * float * \ldots * float \to float
\end{gather*}
Hvor typekonstruktøren "$*$" svarer til det kartesiske produkt, og refereres til som en tuple. I begge tilfælde skal senere definitioner og funktionsanvendelser overholde disse beskrivelser. 

% Derudover kan en type også bestå af en tuple af typer:
% \begin{gather*}
%     f: T_1 * T_2 * \ldots * T_n \rightarrow T_{n+1} \\
%     f(e_1, e_2, \ldots, e_n) : T_{n+1} \iff e_1 : T_1 \land e_2 : T_2 \land \ldots \land e_n : T_n
% \end{gather*}
% En tuple, der kun består af to typer, kaldes et par. Givet en funktion $g: T_1 \rightarrow T_2 \rightarrow T_3$, betyder dette, at den tager et udtryk af typen $T_1$, som giver en funktion af typen $T_2 \rightarrow T_3$, hvor evalueringen af funktionen resulterer i $T_3$. Som eksempel på dette kan vi definere en multivariable funktion $f(x, y) = \sqrt{x! + y!}$ som er en afbildning $f: \mathbb{Z}^2 \to \mathbb{R}$ samt $g(y) = f(3, y) = \sqrt{3! + y!}$, som afbilder $g: \mathbb{Z} \to \mathbb{R}$. De tilsvarende F\# funktioner kan defineres på følgende måder:

% \begin{lstlisting}[language={FSharp}, label={lst:multivariable_function}, caption={Eksempel på typerne for en multivariable funktion i F\#}]
% // f: int -> int -> float
% let f x y = sqrt <| float(factorial x + factorial y)

% // g: int -> float
% let g = f 3
% \end{lstlisting}

En tuple, der kun består af to typer, kaldes et par. Givet en funktion $f: T_1 \rightarrow T_2 \rightarrow T_3$, betyder dette, at den tager et udtryk af typen $T_1$, som giver en funktion af typen $T_2 \rightarrow T_3$, hvor evalueringen af funktionen resulterer i $T_3$.

Hvis en funktion kaldes med et argument, der ikke matcher funktionens type, genereres en fejlmeddelelse. For F\#'s vedkommende kontrolleres dette på oversættelsestidspunktet, hvilket er en af de store forskelle mellem F\# og Python, hvor typerne først kontrolleres under kørslen.

\subsubsection{Induktivt definerede typer}
I F\# defineres lister og træer induktivt ved brug af rekursive typer. Et eksempel på en rekursivt defineret type for et træ er givet i Listing \ref{lst:tree_ex}.

\begin{lstlisting}[language={FSharp}, label={lst:tree_ex}, caption={Eksempel på en rekursivt defineret type for et træ i F\#}]
type Tree = 
    | Leaf of int
    | Node of Tree * Tree
\end{lstlisting}

Rekursive typer lagres som en værdi og ikke en datastruktur.


\subsection{Property Based Testing}
Property Based Testing (PBT) er en teknik til at teste korrekthed af egenskaber som man ved altid skal være opfyldt. Ved PBT genereres en række tilfældige input til en funktion, hvorefter det kontrolleres, om en given egenskab holder. Fokusset ved PBT er de fundamentale egenskaber, som en funktion skal overholde, og ikke de specifikke tilfælde som ved eksempelvis unit tests\footcitetitle{unit_test}.

På DTU introduceres de studerende til logik, som det første emne i "01001 Matematik 1a". Her lærer de at en udsagnslogisk formel er gyldig (en tautologi), hvis den altid er sand. Der findes mange teknikker til at påvise gyldigheden af en udsagnslogisk formel. I "01001 Matematik 1a" anvendes blandt andet sandhedstabeller, som demonstrerer gyldigheden af en formel. Eksempelvis vises hvordan Formel \eqref{udsagnslogik} er gyldig.
\begin{gather}
    P \land (Q \land Y) \iff (P \land Q) \land Y
    \label{udsagnslogik}
\end{gather}

Vi kan også bruge PBT til at undersøge, om Formel \eqref{udsagnslogik} holder, ved at definere egenskaben som en funktion af $P, Q$ og $Y$, som vist i Listing \ref{valid1}.

\lstinputlisting[
    language=FSharp,
    label={valid1},
    caption={PBT af Formel \eqref{udsagnslogik}. Begge sider skal være omgivet af parenteser, for at den er gyldig, da "$=$" har en højere præcedens end" $\&\&$"}]{exampleCodes/valid1.fsx}


\begin{lstlisting}[style=output, label={lst:output_example}, caption={Output ved PBT af (\ref{udsagnslogik})}]
> let _ = Check.Quick propositional_formula;;
Ok, passed 100 tests.
\end{lstlisting}

"Check.Quick" er en del af "FsCheck"-biblioteket. Den tager en funktion som argument og genererer en række tilfældige input til funktionen baseret på funktionens type. Hvis funktionen returnerer "true" for alle input, vil testen lykkes. Hvis funktionen returnerer "false" for et input, vil testen fejle og give et eksempel på et input der fejlede. I Formel \ref{valid1} er "Check.Quick" anvendt til at teste, om Formel \eqref{udsagnslogik} er gyldig. Funktionen "Check.Quick" returnerer "Ok, passed 100 tests," hvilket indikerer, at (\ref{udsagnslogik}) er gyldig. Det er vigtigt at forstå, at dette ikke er det samme som at bevise, at Formel \eqref{udsagnslogik} er gyldig, da testen ikke garanterer, at alle muligheder er testet. Generelt vil der være flere kombinationer af argumenter til en funktion, end de 100 test cases som genereres.


Vi kan også teste den ikke gyldige funktion \textcolor{red}{propositional\_formula\_invalid}, som vist i Listing \ref{output_invalid}, er den ikke gyldig og vi får givet et eksempel på en input kombination der fejlede.

\begin{lstlisting}[style=output, label={output_invalid}, caption={Output ved PBT af \textcolor{red}{propositional\_formula\_invalid}}]
> let _ = Check.Quick propositional_formula_invalid;;
Falsifiable, after 2 tests (0 shrinks) (StdGen (68724885, 297333127)):
Original:
false
true
false
\end{lstlisting}

I nogle tilfælde vil det være en fordel at opskrive en PBT før implementeringen af en funktion, som man ved skal overholde en egenskab. På den måde anvendes Test-Driven Development (TDD)\footcitetitle{TDD} til at teste, om ens egenskab forbliver overholdt under implementeringen. I dette projekt vil vi anvende PBT til at validere, at de matematiske egenskaber bliver overholdt af programmet.



\subsection{Overskrivning af operatorer}
I F\# er det muligt at overskrive standardoperatorer, så de kan anvendes på egne typer. Denne teknik vil blive benyttet igennem rapporten til at definere matematiske operationer for de typer, vi udvikler. Et eksempel på overskrivning af operatorer er givet i næste sektion.

\subsection{Signaturfiler og implementeringsfiler}
En implementeringsfil i F\# er lavet med .fs extension. Implementeringsfilen kan have en tilhørende signaturfil med .fsi extension. Denne fil indeholder en beskrivelse af de typer og funktioner i implementeringsfilen, som er tilgængelige for andre filer. En signaturfil kan derfor bruges som en arbejdstegning for andre, der ønsker at anvende eller replicere implementeringsfilen. Til sidst er der scriptfiler med .fsx extension, som kan bruges til at køre F\# kode uden at skulle kompilere den.

Som et eksempel på de forskellige filtyper kan vi se signaturfilen i Listing \ref{lst:signatur_implementation}.

\lstinputlisting[language={FSharp}, label={lst:signatur_implementation}, caption={Eksempel på en signaturfil \textit{MyInt.fsi}}]{exampleCodes/MyInt.fsi}

Ud fra signaturfilen kan vi se, at implementeringsfilen i Listing \ref{lst:MyInt_fs} indeholder en type \textit{MyInt}, hvor der er defineret overskrivning af operatorerne multiplication og subtraction. Derudover er der defineret en funktion \textcolor{red}{factorial}. 

\lstinputlisting[language={FSharp}, label={lst:MyInt_fs}, caption={Eksempel på en implementeringsfil \textit{MyInt.fs}}]{exampleCodes/MyInt.fs}

I denne implementeringsfil finder vi overskrivningerne af operatorerne på typen \textit{MyInt}. Derudover er der også en implementering af en funktion \textcolor{red}{factorial}, som virker med typen \textit{MyInt}. 

Til sidst kan vi åbne modulet i en scriptfil, som vist i Listing \ref{lst:MyInt_fsx}.

\lstinputlisting[language={FSharp}, label={lst:MyInt_fsx}, caption={Eksempel på en scriptfil \textit{MyInt.fsx}}]{exampleCodes/MyInt.fsx}

Vi kan derefter køre scriptfilen og få outputtet, som vist i Listing \ref{lst:output_test}.

\begin{lstlisting}[style=output, label={lst:output_test}, caption={Output ved kørsel af \textit{MyInt.fsx}}]
I 720
\end{lstlisting}

\subsection{Ræsonnering omkring korrekthed}
Ræsonnering omkring korrekthed i funktionel programmering indebærer flere begreber, der hjælper med at sikre, at programmer opfører sig som forventet. Et program "virker" først, når det er korrekt med hensyn til dets specifikationer\footcitetitle[Kapitel 6]{ML_reason}.

Funktionsprogrammering har \textbf{Referential Transparency}, hvilket betyder, at en funktion altid vil producere det samme output givet det samme input uden bivirkninger. Dette gør koden mere forudsigelig og lettere at teste, da udtryk kan erstattes af deres værdier uden at ændre programmets adfærd\footcitetitle{Referential_transparency}. \textbf{Lambda calculus} er blandt andet med til at give funktionsprogrammering "Referential Transparency".

F\# har et \textbf{stærkt typesystem}, som sikrer, at funktioner ikke kan kaldes med argumenter af forkerte typer. Dette bliver håndteret på kompileringstidspunktet og reducerer dermed mængden af køretidsfejl.

\textbf{Induktivt definerede typer og rekursive funktioner} gør det muligt at bevise korrektheden af programmer ved hjælp af matematiske beviser. For eksempel kan rekursive funktioner bevises korrekte ved hjælp af induktionsbeviser\footcitetitle[Kapitel 6]{ML_reason}.

Disse koncepter kombineret giver et stærkt grundlag for at skrive pålidelig og korrekt software, hvor mange fejl kan fanges tidligt i udviklingsprocessen.



