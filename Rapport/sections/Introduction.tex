\section{Introduktion} 

Dette projekt fokuserer på funktionel modellering af matematiske systemer ved brug af programmeringssproget F\#. I en tid, hvor programmeringssprog som Python dominerer i tekniske og videnskabelige miljøer, undersøger dette projekt potentialet og fordelene ved funktionel programmering i matematiske sammenhænge. I 2023 valgte Danmarks Tekniske Universitet at anvende Python som et hjælpeværktøj i deres grundlæggende matematikkursus "01001 Matematik 1a (Polyteknisk grundlag)". Imidlertid åbner funktionel programmering op for et andet perspektiv og metoder, som kan berige og muligvis forbedre forståelsen af matematiske koncepter hos studerende.

Projektet har til formål at demonstrere, hvordan funktionel programmering, specifikt gennem F\#, kan anvendes til at opbygge og manipulere matematiske udtryk og systemer. Ved at introducere læserne til grundlæggende såvel som avancerede funktioner og teknikker i F\#, vil rapporten guide dem gennem opbygningen af funktionelle programmer, der kan løse matematiske problemer. 

Rapporten vil først og fremmest dykke ned i konstruktionen af et specifikt modul for håndtering af symbolske matematiske udtryk og matrix manipulering, og deres anvendelser i forskellige matematiske kontekster. Projektets struktur og metodologi har til formål at give læseren en dybdegående forståelse af, hvordan funktionel programmering kan benyttes strategisk i matematiske discipliner, og hvordan det adskiller sig fra mere traditionelle imperative programmeringstilgange.

Gennem en systematisk tilgang til design og implementering af matematiske moduler vil rapporten udforske, hvordan matematiske og logiske principper kan integreres direkte i softwareudvikling gennem funktionel programmering. Dette vil ikke kun fremme en bedre forståelse af teoretiske koncepter gennem praktisk anvendelse, men også demonstrere F\#'s kapacitet og effektivitet i behandlingen af matematiske egenskaber.
 