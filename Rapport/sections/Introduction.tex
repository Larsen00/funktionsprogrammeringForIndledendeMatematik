% \section{Introduktion} 
% I 2023 besluttede Danmarks Tekniske Universitet (DTU) at anvende Python som et hjælpemiddel, der erstattede det matematiske softwareprogram Maple\footcitetitle{maple} i deres grundlæggende matematikkurser "01001 Matematik 1a" og "01002 Matematik 1b". I en tid, hvor programmeringssprog som Python dominerer i tekniske og videnskabelige miljøer\footcitetitle{python_pop}, undersøger dette projekt fordele og ulemper ved funktionel programmering i matematiske sammenhænge. Funktionel programmering introducerer et andet perspektiv og alternative metoder, som kan berige og muligvis forbedre de studerendes forståelse af matematiske koncepter.

% Projektet har til formål at belyse, hvorvidt funktionel programmering, specifikt gennem F\#, kan anvendes til at opbygge og manipulere matematiske udtryk og systemer der introduceres i den indledende matematiske undervisnig. Ved at introducere læserne til grundlæggende såvel som avancerede funktioner og teknikker i F\#, vil rapporten guide dem gennem opbygningen af funktionel modellering af matematiske systemer, der kan løse matematiske problemer fra de grundlæggende matematik kurser på DTU. 

% Rapporten vil først og fremmest dykke ned i konstruktionen af et specifikt modul for håndtering af symbolske matematiske udtryk og matrix manipulering, og deres anvendelser i forskellige matematiske kontekster. Projektets struktur og metodologi har til formål at give læseren en dybdegående forståelse af, hvordan funktionel programmering kan benyttes strategisk i matematiske discipliner, og hvordan det adskiller sig fra mere traditionelle imperative programmeringstilgange.

% Gennem en systematisk tilgang til design og implementering af matematiske moduler vil rapporten udforske, hvordan matematiske og logiske principper kan integreres direkte i softwareudvikling gennem funktionel programmering. Dette vil ikke kun fremme en bedre forståelse af teoretiske koncepter og konstruktioner gennem praktisk anvendelse, men også demonstrere F\#'s kapacitet og effektivitet i behandlingen af matematiske egenskaber.
 
\section{Introduktion}

I 2023 besluttede Danmarks Tekniske Universitet (DTU) at anvende Python som et hjælpemiddel, der erstattede det matematiske softwareprogram Maple\footcitetitle{maple} i deres grundlæggende matematikkurser "01001 Matematik 1a" og "01002 Matematik 1b". I en tid, hvor programmeringssprog som Python dominerer i tekniske og videnskabelige miljøer\footcitetitle{python_pop}, undersøger dette projekt fordele og ulemper ved funktionel programmering i matematiske sammenhænge. Funktionel programmering introducerer et andet perspektiv og alternative metoder, som kan berige og muligvis forbedre de studerendes forståelse af matematiske koncepter.

Projektet har til formål at belyse, hvorvidt funktionel programmering, specifikt gennem F\#\footcitetitle{fsharp}, kan anvendes til at opbygge og manipulere matematiske udtryk og systemer, der introduceres i den indledende matematiske undervisning. Ved at introducere læserne til grundlæggende såvel som avancerede funktioner og teknikker i F\#, vil rapporten guide dem gennem opbygningen af funktionel modellering af matematiske systemer, der kan løse matematiske problemer fra de grundlæggende matematikkurser på DTU.

Rapporten vil først og fremmest dykke ned i konstruktionen af et specifikt modul for håndtering af symbolske matematiske udtryk og matrixmanipulering, samt deres anvendelser i forskellige matematiske kontekster. Projektets struktur og metodologi har til formål at give læseren en dybdegående forståelse af, hvordan funktionel programmering kan benyttes strategisk i matematiske discipliner, og hvordan det adskiller sig fra mere traditionelle imperative programmeringstilgange.

Gennem en systematisk tilgang til design og implementering af matematiske moduler vil rapporten udforske, hvordan matematiske principper kan integreres direkte i softwareudvikling gennem funktionel programmering. Dette vil ikke kun fremme en bedre forståelse af teoretiske koncepter og konstruktioner gennem praktisk anvendelse, men også demonstrere F\#'s kapacitet og effektivitet i behandlingen af matematiske egenskaber.
