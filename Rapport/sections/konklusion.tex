\section{Konklusion}

Det primære fokuspunkt for rapporten er, at læseren skal opnå en bedre forståelse af matematiske koncepter gennem opbygning af et matematisk program ved hjælp af funktionsprogrammering. Flere studerende oplever, at rekursive funktioner generelt er et svært koncept, og derudover mangler nogle en forståelse for programdesign, for eksempel ved at genbruge kode frem for at ligge den i metoder. Ved at introducere de studerende til rekursivitet gennem et funktionelt programmeringssprog som F\# ved hjælp af matematiske koncepter, som de møder i kurserne "01001 Matematik 1a" og "01002 Matematik 1b", vil det gavne de studerendes forståelse af både matematikken og programmering. Derudover vil kendskabet til funktionsprogrammering forbedre de studerendes evne til at designe bedre programmer også i andre sprog som Python.

Selvom syntaksen for F\# kan være sværere at lære for nogle, er gevinsten ved at lære den markant. Implementeringen af dette program, fra hvordan man udfører operationer af komplekse tal til matrixmanipulation, vil gavne forståelsen af matematikken. De studerende vil forstå, hvorfor diverse funktioner virker, hvilket ikke altid er tilfældet i Python, hvor mange funktioner kan håndteres som en sort boks.

Konklusionen er, at F\# måske ikke vil kunne erstatte Python i matematikkurserne på DTU, men det vil kunne være et godt supplement til at forbedre de studerendes forståelse af matematikken og programmering.
