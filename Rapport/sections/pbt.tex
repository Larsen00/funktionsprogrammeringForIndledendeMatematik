\section{PBT af programmet} 
\subsection{PBT af udtryk}
Før vi begynder at udføre PBT på alle egenskaber vedrørende udtryk, skal vi først bygge en generator for vores talmængde og udtryk.

Vi begynder med at definere en række generatorer, som kan generere blade i vores udtrykstræer i Listing \ref{generator1}.


\begin{lstlisting}[
    language={FSharp}, 
    label={generator1}, 
    caption={Generatorene anvendt til PBT af udtryk}
    ]
let max = 3
let min = -3

// noneZeroGen: Gen<int>
let noneZeroGen = 
    Gen.oneof [ 
        Gen.choose(1, max) ;
        Gen.choose(min, -1)]

// numberGen: Gen<Number>
let numberGen =
    Gen.oneof [
        Gen.map2 (fun x y -> newRational(x, y) |> Rational |> tryReduce ) (Gen.choose(min, max)) noneZeroGen;
        Gen.map (fun x -> Int x) (Gen.choose(min, max));
        Gen.map4 (fun a b c d -> newComplex (newRational(a, b), newRational(c, d)) |> Complex |> tryReduce ) (Gen.choose(min, max)) noneZeroGen (Gen.choose(min, max)) noneZeroGen]

// numberInExprGen: Gen<Expr<Number>>
let numberInExprGen = 
    Gen.map (fun x -> N x) numberGen

// randomListElement: list<'a> -> Gen<'a>
let randomListElement xlist = 
    gen { let! i = Gen.choose(0, List.length xlist - 1)
        return xlist.[i] }

// variableGen: list<char> -> Gen<Expr<'a>>
let variableGen xlist = Gen.map X (randomListElement xlist)

// leafGen: list<char> -> Gen<Expr<Number>>
let leafGen xlist =
    if xlist <> [] then
        Gen.oneof [numberInExprGen; variableGen xlist]
    else
        numberInExprGen

// onlyIntleafGen: list<char> -> Gen<Expr<Number>>
let onlyIntleafGen xlist :Gen<Expr<Number>> = 
    if xlist <> [] then
        Gen.oneof [Gen.map (fun x -> N <| Int x) (Gen.choose(-10, 10)); variableGen xlist]
    else
        Gen.map (fun x -> N <| Int x) (Gen.choose(-10, 10))

// charsSeqGen: char -> char -> seq<Gen<char>>
let charsSeqGen c1 c2 = seq { for c in c1 .. c2 do
                                yield gen { return c} }

// charGen: Gen<char>
let charGen = gen { return! Gen.oneof (charsSeqGen 'A' 'Z')}

// smallEnvGen: Gen<Map<char, Number> * list<char>>
let smallEnvGen =
    gen { 
        let! i = Gen.choose (0, 5)
        let! xlist = Gen.listOfLength i charGen
        let! ns = Gen.listOfLength i numberGen
        return (Map.ofList (List.zip xlist ns), xlist) }

// exprGen: 'a -> int -> ('a -> Gen<Expr<'b>>) -> Gen<Expr<'b>> 
let rec exprGen xlist n leafType = 
    if n = 0 then
        leafType xlist
    else
        Gen.oneof [
            // leaf occurs twice becourse leaf is X or N giving the same probability for each expression 
            leafType xlist; 
            leafType xlist;
            Gen.map2 (fun x y -> Add (x, y)) (exprGen xlist (n/2) leafType) (exprGen xlist (n/2) leafType);
            Gen.map2 (fun x y -> Mul (x, y)) (exprGen xlist (n/2) leafType) (exprGen xlist (n/2) leafType);
            Gen.map2 (fun x y -> Div (x, y)) (exprGen xlist (n/2) leafType) (exprGen xlist (n/2) leafType);
            Gen.map2 (fun x y -> Sub (x, y)) (exprGen xlist (n/2) leafType) (exprGen xlist (n/2) leafType);            
            Gen.map (fun x -> Neg x) (exprGen xlist (n/2) leafType)]


type SmallEnv = Map<char, Number> * char list
type SmallEnvGen =
    static member SmallEnv() =
        {new Arbitrary<SmallEnv>() with
            override _.Generator = smallEnvGen
            override _.Shrinker _ = Seq.empty}

type NumberGen =
static member Number() =
    {new Arbitrary<Number>() with
        override _.Generator = numberGen
        override _.Shrinker _ = Seq.empty}    
\end{lstlisting}

Vi begynder med at definere to variable, som alle funktioner har til rådighed, \texttt{max} og \texttt{min}, som definerer intervallet for de heltal, der kan anvendes i genereringen af Numbers. 
Den første funktion, \textcolor{red}{noneZeroGen}, er en generator for et tilfældigt heltal fra sættet $S_1$:
\begin{gather*}
    S_1 = \{ x \mid x \in \mathbb{Z} \setminus \{0\}, \quad \texttt{min} \leq x \leq \texttt{max} \}.
\end{gather*}
Vi kan dermed bruge \textcolor{red}{noneZeroGen}, da vi ikke ønsker at generere rationale tal med nævneren 0. For at definere en generator for Numbers i form af \textcolor{red}{numberGen}, som genererer et gyldigt Number fra sættet $S_5$:
\begin{align*}
    S_2 &= \left\{ x \mid x \in \mathbb{Z}, \quad \texttt{min} \leq x \leq \texttt{max} \right\} \\
    S_3 &= \left\{ \frac{x}{y} \mid x \in S_1, y \in S_2 \right\} \\
    S_4 &= \left\{ x + yi \mid x, y \in S_3 \right\} \\
    S_5 &= S_2 \cup S_3 \cup S_4
\end{align*}

Funktionen \textcolor{red}{numberInExprGen} er en generator, som ved brug af \textcolor{red}{numberGen} konverterer et \texttt{Number} til et \texttt{Expr<Number>}. Det er disse generatorer, der anvendes til at generere tal til vores talmængde.

Dernæst kommer nogle generatorer til generering af variable. Først har vi \textcolor{red}{randomListElement}, som tager en liste af en vilkårlig type og udvælger et tilfældigt element fra listen. Derudover har vi \textcolor{red}{variableGen}, som tager en liste af karakterer og bruger \textcolor{red}{randomListElement} til at udvælge en af karaktererne og konvertere den til et \texttt{Expr<Number>}.

Dermed er det nu muligt at lave en generator, som kan generere enten et tal eller en variabel fra konstruktørerne af \texttt{Expr<Number>}. Disse to konstruktører er også bladene i vores udtrykstræer. \textcolor{red}{leafGen} genererer et tilfældigt blad i vores udtrykstræ ud fra en liste af karakterer. \textcolor{red}{onlyIntleafGen} fungerer ud fra det samme princip, men genererer kun heltal fra intervallet -10 til 10.

Funktionerne \textcolor{red}{charGen} og \textcolor{red}{charsSeqGen} er generatorer, som sammen genererer en tilfældig karakter mellem 'A' og 'Z'. Endelig har vi en miljøgenerator, \textcolor{red}{smallEnvGen}, som genererer et miljø, der indeholder variable og deres tilsvarende værdier.

Den sidste generator, \textcolor{red}{exprGen}, vil i vores PBT tage en liste af variable, som den må generere blade ud fra, samt hvilken generator der skal anvendes til at generere bladene. Derudover vil den maksimale dybde af udtrykket være \(\log_2(n)\).

Til sidst defineres tre typer, hvor \texttt{SmallEnvGen} og \texttt{NumberGen} gør det muligt inden kørsel af vores PBT at registrere generatorerne.

Vi har dermed nu lavet fundamentet, til at kunne teste vores egenskaber vedrørende udtryk.

\subsubsection{Tal modulet}\label{sec:PBT_number}
De 6 egenskaber fra \ref{egenskab:tal} kan nu testes med PBT. Egenskaberne kan oversættes direkte til funktioner, grundet de overskrivninger vi har lavet på number typen i Listing \ref{pbt:number}. 

\lstinputlisting[
    language=FSharp, 
    caption={\textit{numberPBT.fsx} - funktioner til test af egenskaberne i \ref{egenskab:tal}},
    label={pbt:number}
    ]{../PBT/numberPBT.fsx}
I \textcolor{red}{inverseMultiplicative} anvender vi "Prop.classify" til at tillade, at egenskaben kan slå bestemte fejl. I dette tilfælde tillades det, at der opstår en "DivideByZeroException". Dette sker, fordi generatoren godt kan generere 0, som der ikke kan tages en invers af. For at teste ovensånde funktioner, køres \textit{.fsx} filen, og outputtet kan ses i Listing \ref{output:number}.
\begin{lstlisting}[
    style=output, 
    label={output:number}, 
    caption={Outputtet fra PBT af Number typer, ved kørsel af Listing \ref{pbt:number}}
    ]        
Ok, passed 100 tests.
Ok, passed 100 tests.
Ok, passed 100 tests.
Ok, passed 100 tests.
Ok, passed 100 tests.
Ok, passed 100 tests.
94% PropertyHolds.
6% DivideByZeroExceptions.
\end{lstlisting}

Dermed viser testen at alle egenskaberne fra \ref{egenskab:tal} holder. 

\subsubsection{Homomorfisme af evaluering}\label{sec:PBT_eval_homomorphism}
\subsubsection{Invers morphism mellem infix og prefix}\label{sec:PBT_infix_prefix}
\subsubsection{Simplifikation af udtryk}\label{sec:PBT_simplification}
\subsubsection{Differentiering af udtryk}\label{sec:PBT_differentiation}
\subsection{PBT af vektorer og matricer}

\subsubsection{PBT af matrix operationer}
Det er nu muligt at opstille nogle PBT af der sikre at matricerne overholder matematiske egenskaber i sætning \ref{vector_space_axioms}. Først defineres en generator for matricer, som generere matricer med tilfældige tal fra vores talmængde \ref{number_type}.

\begin{lstlisting}[
    language={FSharp}, 
    label={generators}, 
    caption={Generatorene anvendt til PBT af matrix operationer i }
    ]
// vectorGen : int -> Gen<Vector>
let vectorGen n =
    Gen.listOfLength n numberGen |> Gen.map (fun x -> vector x)

// matrixGen : Gen<Matrix>
let matrixGen =
    gen {
        let! row = Gen.choose(1, 6)
        let! col = Gen.choose(1, 6)
        let! vectors = Gen.listOfLength col (vectorGen row)
        return matrix vectors
    }

type MaxtrixGen =
    static member Matrix() =
        {new Arbitrary<Matrix>() with
            override _.Generator = matrixGen
            override _.Shrinker _ = Seq.empty}

type NumberGen =
    static member Number() =
        {new Arbitrary<Number>() with
            override _.Generator = numberGen
            override _.Shrinker _ = Seq.empty}   
\end{lstlisting}


Vi kan dermed nu lave definere egenskaberne fra \ref{vector_space_axioms} som nogle funktioner, og teste dem med PBT.

\begin{lstlisting}[
    language={FSharp}, 
    label={lst:vector_space_axioms}, 
    caption={Egenskaberne fra sætning \ref{vector_space_axioms} som funktioner}
    ]
vectorCom : Matrix -> bool
let vectorCom m =
    sumRows m = sumRows (flip m)

vectorScalarAss : Matrix -> Number -> Number -> bool
let vectorScalarAss (m:Matrix) (n1:Number) (n2:Number) =
    n1 * (n2 * m) = (n1 * n2) * m

vectorAssCom : Matrix -> Number -> bool
let vectorAssCom m (c:Number) =
    c * (sumRows m) = sumRows (c * m)
\end{lstlisting}

\begin{lstlisting}[
    style=output, 
    label={lst:vector_space_axioms_pbt}, 
    caption={Outputtet fra PBT af vektor Listing \ref{lst:vector_space_axioms}}
    ]
- Arb.register<MaxtrixGen>()
- Arb.register<NumberGen>()
- let _ = Check.Quick vectorCom
- let _ = Check.Quick vectorScalarAss
- let _ = Check.Quick vectorAssCom;;
Ok, passed 100 tests.
Ok, passed 100 tests.
Ok, passed 100 tests.
\end{lstlisting}

\subsubsection{PBT af Gram-Schmidt}\label{sec:pbt_gram_schmidt}
Udfrodringen ved at lave en PBT af Gram-Schmidt er at vektorsættet skal være lineært uafhængige. Derfor laves der en generator som ved at udføre tilfældige række operationer på en diagonal matrix, kan generere en matrix med lineært uafhængige vektorer. 
\#TODO : Lav et bevis for det beholder enskaben for lineært uafhængighed.

\begin{lstlisting}[
    language={FSharp}, 
    label={generators_gram_schmidt}, 
    caption={Generatorene anvendt til PBT af Gram-Schmidt}
    ]
// getBacismatrixGen : int -> Gen<Matrix>
let getBacismatrixGen n =
    Gen.map (fun x -> standardBacis x) (Gen.choose (2, n))

// performRowOperationGen : Matrix -> Gen<Matrix>
let performRowOperationGen m =
    let (D(n, _)) = dimMatrix m
    gen { 
        let! i = Gen.choose(1, n)
        let! j = match i with
            | 1 -> Gen.choose(2, n)
            | _ when i = n -> Gen.choose(1, n-1)
            | _ -> Gen.oneof [
                    Gen.choose(1, i-1); 
                    Gen.choose(i+1, n)]
        let! a = numberGen
        return rowOperation i j a m }


// multipleRowOperationsGen : Matrix -> int -> Gen<Matrix>
let rec multipleRowOperationsGen m count =
    if count <= 0 then Gen.constant m
    else
        gen {
            let! newMatrix = performRowOperationGen m
            return! multipleRowOperationsGen newMatrix (count - 1)
        }

// getIndependetBacisGen : Gen<Matrix>
let getIndependetBacisGen =
    gen { 
        let! m = getBacismatrixGen 5
        let! numberOfOperations = Gen.choose(1, 10)
        let! span = multipleRowOperationsGen m numberOfOperations
        return span }

type IndependetBacis = Matrix
type IndependetBacisGen =
    static member IndependetBacis() =
        {new Arbitrary<Matrix>() with
            override _.Generator = getIndependetBacisGen
            override _.Shrinker _ = Seq.empty}
\end{lstlisting}

Listing \ref{generators_gram_schmidt} viser de forskellige generatorer, som anvendes til PBT (Property-Based Testing) af Gram-Schmidt-processen. Først genereres en tilfældig basis matrix. Dernæst udvælges to tilfældige rækker, \(i\) og \(j\), hvorefter der udføres en rækkeoperation på \(R_j\), således at \(R_j \leftarrow R_j - aR_i\), hvor $a$ er et tilfældigt Number. Denne proces gentages et tilfældigt antal gange.

Dernæst skal vi bruge en funktion til at tjekke om en matrix er en ortogonal basis. \textcolor{red}{isOrthogonalBacis} i Listing \ref{check_orthogonal_basis} tjekker om alle vektorerne i en matrix er ortogonale, ved at tjekke om søjle $v_i$ er ortogonal med $v_{i+1}$, for alle $i \in [1, n-1]$ hvor $n$ er længden på søjlerne. To søjler er ortogonale hvis deres indreprodukt er 0.

\begin{lstlisting}[
    language={FSharp}, 
    label={check_orthogonal_basis}, 
    caption={Funktion til at tjekke om søjlerne i en matrix er en ortogonal basis}
    ]
// isOrthogonalBacis : Matrix -> bool
let rec isOrthogonalBacis (M(vl, o)) =
    if not <| corectOrderCheck (M(vl, o)) C 
    then isOrthogonalBacis <| correctOrder (M(vl, o)) C
    else
    match vl with
    | [] -> true
    | _::[] -> true
    | v::vnext::vrest -> innerProduct v vnext = zero && isOrthogonalBacis (M(vnext::vrest, o))
\end{lstlisting}
%TODO: ET bevis for denne kunne være nice

PB testen \textcolor{red}{gramSchmidtIsOrthogonal} bliver derfor blot at tjekke om en matrix bestående af lineært uafhængige vektorer, der udspænder et underrum, er orthogonale efter Gram-Schmidt processen er blevet anvendt. Grundet tilfældige matematiske operationer, opstår der en støre mængde opstå overflow fejl, derfor godtages disse men klassificeres som overflow.

\begin{lstlisting}[
    language={FSharp}, 
    label={pbt_gram_schmidt}, 
    caption={PBT af Gram-Schmidt processen}
    ]
let gramSchmidtIsOrthogonal (m:IndependetBacis) =
    let res =
        try 
            if orthogonalBacis m |> isOrthogonalBacis then 1 else 0
        with
            | :? System.OverflowException -> 2
    (res = 1 || res = 2)
    |> Prop.classify (res = 1) "PropertyHolds"
    |> Prop.classify (res = 2) "OverflowException"
\end{lstlisting}

\begin{lstlisting}[
    style=output,
    label = {output_gram_schmidt},
    caption = {Output fra PBT af Gram-Schmidt processen}
]
- Arb.register<IndependetBacisGen>()
- let _ = Check.Quick gramSchmidtIsOrthogonal;;
Ok, passed 100 tests.
69% PropertyHolds.
31% OverflowException.
\end{lstlisting}

Outputtet fra PBT af Gram-Schmidt processen kan ses i Listing \ref{output_gram_schmidt}. Som sædvanligt indikere testen kun korrekthed, men ikke garanteret korrekthed.

